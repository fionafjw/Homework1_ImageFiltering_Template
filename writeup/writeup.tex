%%%%%%%%%%%%%%%%%%%%%%%%%%%%%%%%%%%%%%%%%%%%%%%%%%%%%%%%%%%%%%%%%%%%%
%
% CSCI 1430 Writeup Template
%
% This is a LaTeX document. LaTeX is a markup language for producing
% documents. Your task is to fill out this
% document, then to compile this into a PDF document.
%
% TO COMPILE:
% > pdflatex thisfile.tex
%
% For references to appear correctly instead of as '??', you must run
% pdflatex twice.
%
% If you do not have LaTeX and need a LaTeX distribution:
% - Departmental machines have one installed.
% - Personal laptops (all common OS): www.latex-project.org/get/
%
% If you need help with LaTeX, please come to office hours.
% Or, there is plenty of help online:
% https://en.wikibooks.org/wiki/LaTeX
%
% Good luck!
% James and the 1430 staff
%
%%%%%%%%%%%%%%%%%%%%%%%%%%%%%%%%%%%%%%%%%%%%%%%%%%%%%%%%%%%%%%%%%%%%%
%
% How to include two graphics on the same line:
%
% \includegraphics[\width=0.49\linewidth]{yourgraphic1.png}
% \includegraphics[\width=0.49\linewidth]{yourgraphic2.png}
%
% How to include equations:
%
% \begin{equation}
% y = mx+c
% \end{equation}
%
%%%%%%%%%%%%%%%%%%%%%%%%%%%%%%%%%%%%%%%%%%%%%%%%%%%%%%%%%%%%%%%%%%%%%%%%%%%%%%%%%%%%%%%%%%%%%%%%

\documentclass[11pt]{article}

\usepackage[english]{babel}
\usepackage[utf8]{inputenc}
\usepackage[colorlinks = true,
            linkcolor = blue,
            urlcolor  = blue]{hyperref}
\usepackage[a4paper,margin=1.5in]{geometry}
\usepackage{stackengine,graphicx}
\usepackage{fancyhdr}
\setlength{\headheight}{15pt}
\usepackage{microtype}
\usepackage{times}
\usepackage{booktabs}
\usepackage{amssymb}
\usepackage{enumerate}   


% python code format: https://github.com/olivierverdier/python-latex-highlighting
\usepackage{pythonhighlight}

\frenchspacing
\setlength{\parindent}{0cm} % Default is 15pt.
\setlength{\parskip}{0.3cm plus1mm minus1mm}

\pagestyle{fancy}
\fancyhf{}
\lhead{Homework Writeup}
\rhead{CSCI 1430}
\rfoot{\thepage}

\date{}

\title{\vspace{-1cm}Homework Writeup}


\begin{document}
\maketitle
\vspace{-2cm}
\thispagestyle{fancy}

\section*{Instructions}

\begin{itemize}
  \item This write-up is intended to be `light'; its function is to help us grade your work and not to be an exhaustive report.
  \item Be brief and precise.
  \item Please describe any non-standard or interesting decisions you made in writing your algorithm.
  \item Show your results and discuss any interesting findings.
  \item List any extra credit implementation and its results.
  \item Feel free to include code snippets, images, and equations. Below are useful markup templates for these.
  \item \textbf{Please make this document anonymous.}
\end{itemize}

\newpage
% ------------------------------------------------ %

\section*{Declaration of Generative AI Use}

\subsection*{Reminder of Course Policy}

\begin{itemize}
    \item The use of GenAI tools (e.g., ChatGPT, Grammarly, Bard) for completing any part of this course is discouraged.
    \item Using these tools is not needed to be successful in the class and could be detrimental to your learning experience.
    \item If you use them, you must cite the tool you used and explain how you used it.
    \item If you do not cite the tool, it is an academic code violation.
    \item We will be using special tools to detect cases of unattributed GenAI use.
\end{itemize}

\subsection*{Student Declaration}

\subsubsection*{Have you used generative AI tools to complete this assignment:}

%%%%%%%%%%%%%% TODO %%%%%%%%%%%%%%%%%%%%

YES $\square$ NO $\blacksquare$ % change answer to \blacksquare

%%%%%%%%%%%%%%%%%%%%%%%%%%%%%%%%%%%%%%%%

\subsubsection*{If you answered YES above, describe what tools you used and what parts of the assignment you used them for below:}

%%%%%%%%%%%%%% TODO %%%%%%%%%%%%%%%%%%%%

\textit{Example: I used ChatGPT to debug my convolution implementation}

%%%%%%%%%%%%%%%%%%%%%%%%%%%%%%%%%%%%%%%%

% ------------------------------------------------ %
\newpage

\section*{Assignment Overview}

The assignment contains two functions:
\begin{enumerate}
    \item my\_imfilter \\
    It takes in an image and a kernel and convolves the two inputs to make a filtered image
    \item gen\_hybrid\_images \\
    It takes in two images and a cutoff frequency. The function takes low frequencies from the first image with a Gaussian blur (where the standard deviation is based off the cutoff frequency), high frequencies from the second image, and combines the filtered images to create a hybrid image.
\end{enumerate}

\section*{Implementation Detail}

In my first approach in writing the function my\_imfilter, I modeled the mathematical equation for 2D convolution,
\begin{center}
$h[m, n] = \sum_{k,l} f[k,l]I[m-k, n-l]$
\end{center}
in code. For gray scale images I used 2 loops to visit every element of the image. At each element, I used logical indexing to extract a matrix of the kernel's shape which is highlighted in this code snippet.
\begin{python}
neighborhood = image[x-pad_h:x+pad_h+1, y-pad_w: y+pad_w+1, c]
\end{python}
The used numpy functions multiply and sum to perform element wise multiplication on the kernel and neighborhood and summing all the elements together. Then I stored the sum into the corresponding coordinate of the output image.\\
For RGB images, the code performs the same process except three times on each channel, so, I implemented a third for loop. \\\\
\textbf{Final Improved Approach to Speed Up}: I realized that this approach was slow and wrote a new function to take advantage of numpy array operations. Instead of updating one pixel at a time, the optimized function performs multiplication by each kernel element, as the kernel size is much smaller than the image size. Then the function shifts each multiplied image by the distance from the kernel center and adds each shifted matrix together.
\\\\
gen\_hybrid\_image first takes low frequencies from image1 with a Gaussian blur and then takes high frequencies from image2 by taking a Gaussian blur on image2 and then subtracting the filtered image2 from the original image2. Then it adds the low frequency image1 and high frequency image2 and clips out of range values to produce the hybrid image.

\section*{Result}
On the next page are my three hybrid images:

\begin{figure} [h]
    \centering
    \includegraphics[width=4cm]{high_frequencies1.jpg}
    \includegraphics[width=4cm]{low_frequencies1.jpg}
    \includegraphics[width=4cm]{hybrid_image1.jpg}
    \caption{\emph{Left:} High frequency image of cat. \emph{Center:} Low frequency image of dog. \emph{Right:} Hybrid image.}
\end{figure}

\begin{figure} [h]
    \centering
    \includegraphics[width=4cm]{high_frequencies2.jpg}
    \includegraphics[width=4cm]{low_frequencies2.jpg}
    \includegraphics[width=4cm]{hybrid_image2.jpg}
    \caption{\emph{Left:} High frequency image of motorcycle. \emph{Center:} Low frequency image of bicycle. \emph{Right:} Hybrid image.}
\end{figure}

\begin{figure} [h]
    \centering
    \includegraphics[width=4cm]{high_frequencies3.jpg}
    \includegraphics[width=4cm]{low_frequencies3.jpg}
    \includegraphics[width=4cm]{hybrid_image3.jpg}
    \caption{\emph{Left:} High frequency image of bird. \emph{Center:} Low frequency image of plane. \emph{Right:} Hybrid image.}
\end{figure}

\section*{Extra Credit (Optional)}
I created my own hybrid image of a cat and my friend. 
\begin{figure} [t]
    \centering
    \includegraphics[width=4cm]{myhigh_frequencies.jpg}
    \includegraphics[width=4cm]{mylow_frequencies.jpg}
    \includegraphics[width=4cm]{myhybrid_image.jpg}
    \caption{\emph{Left:} High frequency image of a cat. \emph{Center:} Low frequency image of my friend. \emph{Right:} Hybrid image.}
\end{figure}

\end{document}
